\documentclass{zusammenfassung}
\usepackage{array}
\usepackage{booktabs}
\usepackage{color}
\usepackage{colortbl}
\graphicspath{ {./illustrationen/} }
\setmonofont{Linux Libertine Mono}

\begin{document}
\maketitle{Klasse 6/7}{10. Februar 2015}{2014/2015}

\begin{aufgabe}
  Wie lautet der Klartext, der zu folgendem mit einer Skytale chiffrierten Geheimtext gehört?

  \texttt{ISADTPIHEHNNCSDIROOILTAIHTAEASNFEZCSWESSNISFIUKTUJSESTCRCKE!}
\end{aufgabe}

\begin{aufgabe}
  Wie viele Permutationen hat ein Alphabet mit 2 Buchstaben, mit 3 Buchstaben, mit 4 Buchstaben und mit 26 Buchstaben?
\end{aufgabe}

\begin{aufgabe}
  Folgender Text ist mit der Cäsar-Verschlüsselung chiffriert worden:

  \texttt{HMI GEIWEV-ZIVWGLPYIWWIPYRK MWX OIMRI WILV WMGLIVI QIXLSHI.}

  Was wurde hier verschlüsselt?
\end{aufgabe}

\begin{aufgabe}
  Knacke folgenden Geheimtext, der mit einer Schlüsselwortchiffrierung mit einem deutschen Schlüsselwort verschlüsselt wurde:

  \texttt{YZBY YZBKJYNGJBO ZB XZY MZHHYBHWNUKI LCA LYGHWNTJYHHYTB, \\LYGVYGOYB JBX LYGNYZATZWNYB CNBY UTTY
  OYNYZABZHRGUYAYGYZ, \\UVYG BZWNI CNBY NZBIYGTZHIZOYB HWNUTR, XUGOYHIYTTI SJA \\BJISYB JBX YGOCYISYB XYH UTTOYAYZBYB DJVTZRJAH.}
\end{aufgabe}

\pagebreak
\begin{aufgabe}
  Der folgende Text wurde mit dem Vigenère-Algorithmus verschlüsselt:

  \begin{center}
    \fontspec{Linux Libertine Mono}
    \begin{tabular}{ccccc}
      PYIPJ&MHQYW&ECJMZ&QXZZD&AGRDT\\
      XUZCW&PYMSY&QHVBW&IUICW&OBJEF\\
      PTNKF&LCXKD&EYIGS&QBIOF&PYZXW\\
      DTNOA&VUVRJ&UAVXE&UMJSG&ZCEBG\\
      YULPV&UYJMZ&DCWDW&ZUCLW&DNZCK\\[1ex]

      FCVCK&MHWKF&SMYKL&FYVBS&GZXBM\\
      ZXJOA&ZYINA&BFFWS&FCJMZ&QHKKW\\
      FCXUW&UNEEJ&BLRUL&UMTRW&ECEDW\\
      DYJCW&MHUOJ&WLPZL&AAIKH&TCVNS\\
      ZHZWS&XNVBN&AHEOM&ZOENV&DYZCK\\[1ex]

      UAAKZ&DYELW&EWYVG&EMMSY&QHVBW\\
      PUJCW&DHLXY&QHLOY&QHUFW&DGFOY\\
      QHVBO&ALSOF&TUSOM&ZXJOA&ZFVLW\\
      ZELOF&RNZQV&QLNSK&EYECU&TUWDO\\
      UXDOF&IICVW
    \end{tabular}
  \end{center}
  \begin{enumerate}
    \item Bestimme mit dem Kasiski-Test mögliche Schlüssellängen.
    \item Bestimme mit dem Friedman-Test die Größenordnung der Schlüssellänge.
    \item Was ist also mit großer Wahrscheinlichkeit die richtige Schlüssellänge?
    \item Versuche, den Text zu entschlüsseln. Was ist das Schlüsselwort und was der Klartext?
  \end{enumerate}
\end{aufgabe}

\end{document}
