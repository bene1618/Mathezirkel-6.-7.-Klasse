\documentclass{zusammenfassung}
\usetikzlibrary{matrix}
\graphicspath{ {./illustrationen/} }

\begin{document}
\maketitle{Klasse 6/7}{09. Dezember 2014}{2014/2015}

Bei der \emph{1-\$-Auktion} wird ein Dollar versteigert. Derjenige, der den größten Betrag geboten hat, erhält wie bei einer 
normalen Auktion gegen sein Gebot den Dollar. Allerdings muss auch derjenige, der am zweitmeisten bietet, seinen gebotenen Betrag 
abgeben.

Die beste Lösung wäre, wenn sich die Teilnehmer darauf einigen würden, dass nur ein einziger einen Cent (also den Mindestbetrag)
bietet, und der Gewinn dann aufgeteilt wird. Im Durchschnitt gehen die Gebote allerdings bis \$3,40.

Beim sogenannten \emph{Ultimatumsspiel} gibt es zwei Spieler A und B. Spieler A hat einen Betrag (etwa einen Euro) zur Verfügung
und muss einen Teil davon Spieler B anbieten. Wenn Spieler B annimmt, darf Spieler A den Rest und Spieler B den angebotenen Betrag
behalten; ansonsten gehen beider leer aus. Da es sich für Spieler B schon lohnen würde, nur einen Cent zu bekommen, müsste bei
vollkommen rationalem Verhalten A einen Cent anbieten und B das Angebot annehmen. Da sich B aber in diesem Fall ungerecht
behandelt fühlen wird, wird er meistens das Angebot eher ablehnen. Das zeigt, dass rationale Entscheidungen nicht immer gute
Modelle für tatsächliches Verhalten sind.

Das \emph{Gefangenendilemma} beschreibt folgende Situation: Zwei Personen A und B sind beide eines Verbrechens beschuldigt, ihnen
können aber nur kleinere Vergehen nachgewiesen werden, bei denen sie jeweils für 2 Jahre ins Gefängnis gehen müssten. Daher wird
beiden das Angebot gemacht, dass sie freikommen, wenn sie durch ihr Geständnis den jeweils anderen belasten. Der müsste dann für 6
Jahre ins Gefängnis gehen. Wenn allerdings beide gestehen, dann wird das Geständnis als mildernder Umstand gesehen und beide
müssten für jeweils 4 Jahre ins Gefängnis. Schematisch lässt sich das in sogenannter \emph{Normalform} darstellen:

\begin{center}
  \begin{tikzpicture}
    \matrix[matrix of nodes] (table) {
      \phantom{gesteht nicht}& gesteht & gesteht nicht \\
      \phantom{gesteht nicht}& 2/2 & 0/6 \\
      gesteht nicht & 6/0 & 4/4 \\
    };
    \node at (table-2-1) {gesteht}; 
    \draw[double distance=0.8pt] (table-1-1.north east) -- (table-3-1.south east);
    \draw[double distance=0.8pt] (table-1-1.south west) -- (table-1-3.south east);
    \node[anchor=south] at ([yshift=0.1cm]$(table-1-2.north)!0.5!(table-1-3.north)$) {Spieler A};
    \node[anchor=east] at ([xshift=-0.1cm]$(table-2-1.west)!0.5!(table-3-1.west)$) {Spieler B};
  \end{tikzpicture}
\end{center}

Die Tabelle bedeutet, dass Spieler A den Wert vor dem Strich und Spieler B den Wert nach dem Strich als "`Auszahlung"' erhält. Die
Auszahlung ist hier die Anzahl der Jahre, die der jeweilige Spieler in den nächsten 6 Jahren in Freiheit verbringen darf, und
jeder Spieler möchte die Auszahlung für sich maximieren. Wenn sich beide Spieler absprechen können, ist es vernünftig, dass keiner
von beiden gesteht, da so die Gesamtauszahlung am höchsten ist. Aus der Sicht jedes einzelnen Spielers jedoch ist es sinnvoller,
nicht zu gestehen, da so die Auszahlung bei jeder Entscheidung des anderen Spielers maximal wird.

Ein \emph{Nash-Gleichgewicht} besteht aus einer Strategie für jeden Spieler, sodass jeder Spieler unabhängig von der gewählten
Strategie der anderen Spieler mit seiner Strategie zufrieden ist. Im Gefangenendilemma ist ein Nash-Gleichgewicht also, wenn beide
Spieler gestehen, denn: Egal, ob ein Spieler gesteht oder nicht, ist für den anderen Spieler die Auszahlung bei einem Geständnis
höher.

Es gibt zwei Arten von Strategien: Eine \emph{reine Strategie} liegt vor, wenn ich als Spieler \emph{eine} bestimmte Entscheidung
treffe. Eine gemischte Strategie bedeutet, dass ich eine zufällige Entscheidung (etwa mit Hilfe eines Würfels) zwischen zwei oder
mehr möglichen Strategien treffe.

Ein Spiel, in dem kein Nash-Gleichgewicht mit reinen Strategien existiert, ist das Spiel \emph{Eisessen vs. Kino}. Hier haben sich
zwei Freunde verabredet, nach der Schule entweder Eis essen oder ins Kino zu gehen. Spieler 1 möchte dabei lieber Eis essen,
Spieler 2 lieber ins Kino gehen. Beide sind aber unglücklich, wenn sie nichts zusammen unternehmen, und es besteht für die beiden
Spieler keine Möglichkeit, sich vorher abzusprechen. Die Normalform des Spiels schaut dann folgendermaßen aus:

\begin{center}
  \begin{tikzpicture}
    \matrix[matrix of nodes] (table) {
      \phantom{Kino}& Eis& Kino\\
      \phantom{Kino}& 3/1 & 0/0 \\
      Kino & 0/0 & 1/3 \\
    };
    \node at (table-2-1) {Eis}; 
    \draw[double distance=0.8pt] (table-1-1.north east) -- (table-3-1.south east);
    \draw[double distance=0.8pt] (table-1-1.south west) -- (table-1-3.south east);
    \node[anchor=south] at ([yshift=0.1cm]$(table-1-2.north)!0.5!(table-1-3.north)$) {Spieler A};
    \node[anchor=east] at ([xshift=-0.1cm]$(table-2-1.west)!0.5!(table-3-1.west)$) {Spieler B};
  \end{tikzpicture}
\end{center}

Hier hängt es für jeden Spieler von der Entscheidung des anderen Spielers ab, durch welche Entscheidung die Auszahlung maximiert
werden kann, also existiert kein Nash-Gleichgewicht aus reinen Strategien. 

Eine ganz andere Art von Spiel ist das sogenannte \emph{Nim-Spiel}. Hier werden Streichhölzer in mehreren Reihen ausgelegt, etwa
folgendermaßen:

\def\line{{\draw (0,-0.1)--(0,0.3);}}
\begin{center}
  \begin{tikzpicture}[every path/.style={thick}]
    \matrix[column sep=3pt,row sep=3pt] {
      \line\\
      \line & \line \\
      \line & \line & \line \\
      \line & \line & \line & \line \\
      \line & \line & \line & \line & \line \\
    };
  \end{tikzpicture}
\end{center}

Die Spieler dürfen nun abwechselnd aus eine Reihe ihrer Wahl beliebig viele Hölzer wegnehmen. Wer das letzte Holz wegnimmt,
gewinnt.

Bei dieser Anfangsposition gewinnt derjenige, der den ersten Zug macht. Um das zu sehen, brauchen wir \emph{Dualzahlen}. So wie
bei Dezimalzahlen
\[
  17438 = 1\cdot \underbrace{10000}_{=10^4}+7\cdot\underbrace{1000}_{=10^3}+4\cdot\underbrace{100}_{=10^2}+3\cdot 10+8\cdot 1
\]
ist, sind Dualzahlen eine Darstellung von Zahlen mit Hilfe von Zweierpotenzen, also beispielsweise
\[
  (10011)_2=1\cdot 2^4+0\cdot 2^3+0\cdot 2^2+1\cdot 2+1=16+2+1=19.
\]

Jetzt schreiben wir die Anzahl jeder Reihe als Dezimalzahl und addieren jeweils die Einer-, Zweier- und Viererstellen:

\begin{center}
  \begin{tikzpicture}[every path/.style={thick}]
    \matrix[column sep=3pt,row sep=3pt] {
      \line & & & & &[0.2cm] \node[anchor=base] {$1=$}; & \node[anchor=base] {$(001)_2$};\\
      \line & \line & & & & \node[anchor=base] {$2=$}; & \node[anchor=base] {$(010)_2$};\\
      \line & \line & \line & & & \node[anchor=base] {$3=$}; & \node[anchor=base] {$(011)_2$};\\
      \line & \line & \line & \line & & \node[anchor=base] {$4=$}; & \node[anchor=base] {$(100)_2$};\\
      \line & \line & \line & \line & \line & \node[anchor=base] {$5=$}; & \node[anchor=base] {$(101)_2$};\\
					       & & & & & & \draw (-0.6,0)--(0.6,0);\\
					       &&&&&& \node[anchor=base] {$223$}; \\
    };
  \end{tikzpicture}
\end{center}

Nun muss man in jedem Zug genau so viele Hölzer entfernen, dass in der Summe unten nur gerade Zahlen vorkommen, also in diesem
Fall ein Holz entweder aus der ersten, der dritten oder der fünften Reihe entfernen. Wenn nun noch Hölzer übrig sind, muss der
nächste Spieler auf jeden Fall eine der Zahlen in der Summe um 1 verringern, also hat der erste Spieler wieder eine ungerade Zahl
vor sich und kann von vorne beginnen.

Beispiel: In folgender Situation muss die 1 ganz links entfernt werden. Diese 1 entspricht 4 Hölzern, also müssen alle vier Hölzer
aus der 4. Reihe weggenommen werden.

\begin{center}
  \begin{tikzpicture}[every path/.style={thick}]
    \matrix[column sep=3pt,row sep=3pt] {
      \line & & & & &[0.2cm] \node[anchor=base] {$1=$}; & \node[anchor=base] {$(001)_2$};\\
      \line & \line & & & & \node[anchor=base] {$2=$}; & \node[anchor=base] {$(010)_2$};\\
      \line & \line & & & & \node[anchor=base] {$2=$}; & \node[anchor=base] {$(010)_2$};\\
      \line & \line & \line & \line & & \node[anchor=base] {$4=$}; & \node[anchor=base] {$(100)_2$};\\
      \line & \line & \line& & & \node[anchor=base] {$3=$}; & \node[anchor=base] {$(011)_2$};\\
					       & & & & & & \draw (-0.6,0)--(0.6,0);\\
					       &&&&&& \node[anchor=base] {$142$}; \\
    };
  \end{tikzpicture}
\end{center}

\end{document}
