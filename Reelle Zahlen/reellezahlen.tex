\documentclass{zusammenfassung}
\usepackage{array}
\usepackage{booktabs}
\usepackage{color}
\usepackage{colortbl}
\usepackage{xstring}
\usepackage{intcalc}
\usepackage{alphalph}
\usepackage[misc]{ifsym}
\usepackage{ziffer}
\graphicspath{ {./illustrationen/} }
\setmonofont{Linux Libertine Mono}
\usetikzlibrary{positioning}
\usetikzlibrary{matrix}

\begin{document}
\maketitle{Klasse 6/7}{21. Juli 2015}{2014/2015}

Gibt es eine Zahl, deren Quadrat gleich $2$ ist, also eine Zahl $x$ mit $x^2=2$ (so eine Zahl nennt man \emph{(Quadrat-)Wurzel} von
$2$ und schreibt sie als $\sqrt 2$)? Falls $x=\frac mn$ ein Bruch ist, also $m$ und $n$ natürliche Zahlen sind, dann folgt
\[
	\frac{m^2}{n^2}=\left(\frac mn\right)^2=2
\]
beziehungsweise (wenn man beide Seiten mit $n^2$ multipliziert)
\[
	m^2=2n^2.
\]
In der Primfaktorzerlegung von $m^2$ kommt die $2$ mit einer geraden Potenz vor, da jede $2$ in der Primfaktorzerlegung von $m$
quadriert wird. Genauso kommt die $2$ in der Primfaktorzerlegung von $n^2$ mit einer geraden Potenz vor, also kommt sie in der
Primfaktorzerlegung von $2n^2$ mit einer ungeraden Potenz vor. Das kann aber nicht sein, weil ja $2n^2=m^2$ ist. Also erfüllt kein
Bruch $x=\frac mn$ die Gleichung $x^2=2$.

Wir können aber versuchen, Annäherungen für $\sqrt 2$ zu finden. Beispielsweise ist $1^2=1$ kleiner als $2$, aber $2^2=4$ größer
als $2$, also muss $\sqrt 2$ zwischen $1$ und $2$ liegen. Genauere Annäherungen sind folgende:
\begin{align*}
	1,4^2=1,96<2,\; 1,5^2=2,25>2&\leadsto \sqrt 2\in[1,4;1,5]\\
	1.41^2=1,9881<2,\;1,42^2=2,0164>2&\leadsto \sqrt 2\in[1,41;1,42]\\
	1,414^2=1,999396<2,\;1,415^2=2,002225>2&\leadsto \sqrt 2\in[1,414;1,415]
\end{align*}
Indem man so weitermacht, kann man $\sqrt 2$ als unendlichen Dezimalbruch, beginnend mit $1,414\ldots$, auffassen.

An der Stelle $\sqrt 2$ haben die rationalen Zahlen (also die Menge aller Brüche) also eine Lücke, die man mit dieser Konstruktion
füllen kann. Wir bezeichnen die Menge aller unendlichen Dezimalbrüche mit $\mathbb R$. Das sind die sogenannten \emph{reellen
Zahlen}. Beispiele für reelle Zahlen sind natürlich alle rationale Zahlen, aber eben auch zum Beispiel $\sqrt 2$. Die reellen
Zahlen, die nicht rational sind, sich also nicht als Bruch schreiben lassen, heißen \emph{irrationale Zahlen}. Wir haben also
gerade gesehen, dass $\sqrt 2$ eine irrationale Zahl ist. Die Aufgaben beschäftigen sich mit weiteren irrationalen Zahlen, wobei
die Anzahl der Sternchen $*$ die Schwierigkeit der Aufgabe angibt: Je mehr Sternchen, desto schwieriger.

\begin{aufgabe}
	Welche der folgenden Zahlen sind rational, lassen sich also als Bruch schreiben?
	\begin{enumerate}
		\item $\sqrt 3$
		\item $\sqrt 4$
		\item $\sqrt p$, wobei $p$ eine Primzahl ist
		\item $\sqrt n$, wobei $n$ keine Quadratzahl ist
	\end{enumerate}
\end{aufgabe}

\begin{aufgabe}[$(*)$]
	Berechne die ersten $3$ Nachkommastellen von $\sqrt 3$ (ohne Taschenrechner)!
\end{aufgabe}

\begin{aufgabe}
	$\sqrt[3]{12}$ ist eine reelle Zahl, die die Gleichung $(\sqrt[3]{12})^3=12$ erfüllt. Ist $\sqrt[3]{12}$ rational?
\end{aufgabe}

\def\phi{\varphi}

\begin{aufgabe}
	\begin{enumerate} 
		\item $(***)$ Überlege dir, dass folgende Aussage wahr ist: \emph{Seien $x$ und $y$ positive reelle Zahlen. Dann hört Euklids
				Algorithmus genau dann auf (er "`terminiert"'), wenn $x$ ein rationales Vielfaches von $y$ ist, also $x=qy$ für eine
			rationale Zahl $q\in\mathbb Q$ gilt.}
		\item $(**)$ Der \emph{goldene Schnitt} $\phi$ ist das Verhältnis $\frac ba$ mit der Eigenschaft, dass $\frac{b+a}b=\phi=\frac
			ba$ ist.
			\begin{center}
				\begin{tikzpicture}[x=4cm]
					\draw (0,-0.2) -- (0,0.2)
					  (1,-0.2) -- (1,0.2)
						(0.618,-0.2) -- (0.618,0.2);
					\draw (0,0) -- (1,0);
					\node[anchor=base] at (0.309,0.15) {b};
					\node[anchor=base] at (0.809,0.15) {a};
			  \end{tikzpicture}
			\end{center}
			Zeige, dass der euklidische Algorithmus mit $b$ und $b+a$ nicht terminiert. Folgere daraus, dass $\phi$ irrational ist.
	\end{enumerate}
\end{aufgabe}

\begin{aufgabe}[$(***)$]
	Sei $x\in\mathbb Q$ eine rationale Lösung der Gleichung
	\begin{equation}
		a_0+a_1x+a_2x^2+\cdots+a_nx^n=0,\tag{$\heartsuit$}
	\end{equation}
	wobei $a_0,a_1,\ldots,a_n\in\mathbb Z$ ganze Zahlen seien. Sei außerdem $x=\frac kl$ eine vollständig gekürzte Darstellung von
	$x$, also $k\in\mathbb Z$, $l\in\mathbb N$ und $\text{ggT}(k,l)=1$. Zeige, dass dann gilt:
	\begin{enumerate}
		\item $k$ ist ein Teiler von $a_0$
		\item $l$ ist ein Teiler von $a_n$
	\end{enumerate}
	\emph{Tipp: Multipliziere die Gleichung ($\heartsuit$) mit $l^n$. $k$ und $l$ sind Teiler der rechten Seite (da steht $0$), also
	müssen sie auch Teiler der linken Seite sein.}
\end{aufgabe}

\begin{aufgabe}[$(**)$]
	Der goldene Schnitt $\phi$ wurde in Aufgabe 4b) definiert.
	\begin{enumerate}
		\item Überlege dir, dass $\phi$ die Gleichung $\phi^2=1+\phi$ erfüllt.
		\item Forme diese Gleichung um, um zu zeigen, dass $\phi$ auch $1+\phi-\phi^2=0$ erfüllt.
		\item Verwende Aufgabe $5$, um zu zeigen, dass $\phi$ irrational ist.
	\end{enumerate}
\end{aufgabe}

\begin{aufgabe}[$(*)$]
	Warum ist das Quadrat jeder reellen Zahl positiv? Kann man also die Gleichung $x^2=-1$ für reelle Zahlen lösen?
\end{aufgabe}

Die letzte Aufgabe zeigt, dass man auch mit reellen Zahlen an gewisse Grenzen stößt, da man nicht jede Gleichung lösen kann. Man
kann über dieses Problem hinwegkommen, indem man einfach eine neue Zahl $i$ einführt, die $i^2=-1$ erfüllt. Die Zahl $i$ kann
nicht auf der Zahlengerade liegen, sondern liegt sozusagen "`neben ihr"'. Eine \emph{komplexe Zahl} ist dann die Summe von
Vielfachen von $i$ und einer reellen Zahl, also etwa eine Zahl wie $\phi+\sqrt 2\cdot i$, wobei $\phi$ der goldene Schnitt aus
Aufgabe 4b) ist. Man bezeichnet die Menge der komplexen Zahlen mit $\mathbb C$. Eine Möglichkeit, sich komplexe Zahlen
vorzustellen, ist die sogenannte \emph{Zahlenebene}:
\begin{center}
	\begin{tikzpicture}
		\draw[->] (-2.5,0) -- (2.5,0);
		\draw[->] (0,-2.5) -- (0,2.5);

		\foreach \x in {-2,-1,1,2}
		  \draw[xshift=\x cm] (0pt,1pt) -- (0pt,-1pt) node[below,fill=white] {$\x$};
		\foreach \y/\ytext in {-2,-1/-,1/,2}
		  \draw[yshift=\y cm] (1pt,0pt) -- (-1pt,0pt) node[left,fill=white] {$\ytext i$};

			\draw[fill=black] (1,2) circle (1pt) node[below] {$1+2i$};
	\end{tikzpicture}
\end{center}

Komplexe Zahlen sind dann Punkte in der Zahlenebene. Quadrieren verdoppelt dann einfach den Winkel zu den positiven reellen
Zahlen: $-1$ hat einen Winkel von $180^\circ$ und wird auf $360^\circ$ (also auf die positiven reellen Zahlen) abgebildet. $i$ hat
einen Winkel von $90^\circ$ und wird auf $180^\circ$, also auf eine negative reelle Zahl (nämlich $-1$) abgebildet. Es stellt sich
heraus, dass man in den komplexen Zahlen nicht nur die Gleichung $x^2=-1$ lösen kann, sondern sogar viel mehr Gleichungen, nämlich
alle Polynomgleichungen
\[
	a_0+a_1x+\cdots+a_nx^n=0,
\]
wobei $n\geq 1$ und $a_n\neq 0$ sein muss. Die Forderung $n\geq 1$ ist wichtig, weil man sonst die Gleichung $a_0=0$ hätte, die
natürlich für $a_0\neq 0$ keine Lösung besitzen kann (die Variable $x$ kommt ja noch nicht einmal vor).
\end{document}
