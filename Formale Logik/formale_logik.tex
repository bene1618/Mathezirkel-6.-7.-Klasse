\documentclass{zusammenfassung}
\usepackage{array}
\usepackage{booktabs}
\usepackage{color}
\usepackage{colortbl}
\graphicspath{ {./illustrationen/} }

\begin{document}
\maketitle{Klasse 6/7}{3. Februar 2015}{2014/2015}

Schauen wir uns folgenden Beweis an: \emph{Auf jeden Fall ist $-2=-2$ und daher auch $4-6=1-3$ sowie $4-6+\frac 94=1-3+\frac 94$.
  Mit der zweiten binomischen Formel erhält man, dass $(2-\frac 32)^2=4-6+\frac 94$ und $(1-\frac 32)^2=1-3+\frac 94$ ist. Also
muss $(2-\frac 32)^2=(1-\frac 32)^2$ gelten. Das ist der Fall, wenn $2-\frac 32=1-\frac 32$ ist. Es folgt $2=1$.}

Was ist falsch an diesem Beweis? (Oder müssen wir akzeptieren, dass $2=1$ ist. Das wäre dann aber noch nicht die schlimmste
Konsequenz: Wenn man jetzt auf beiden Seiten $1$ abzieht, dann erhält man $1=0$. Bei der Multiplikation mit einer beliebigen Zahl
$x$ folgt $x=0$. Also sind alle Zahlen gleich.)

Das Problem ist ein logischer Fehler: Es ist zwar wahr, dass $(2-\frac 32)^2=(1-\frac 32)^2$ ist, wenn $2-\frac 32=1-\frac 32$
gilt. Die Umkehrung stimmt aber nicht: Nur weil $(2-\frac 32)^2=(1-\frac 32)^2$ ist (das ist wahr, denn beide Seiten ergeben 
$\frac 14$), muss nicht $2-\frac 32=1-\frac 32$ sein (auf der linken Seite steht $+\frac 12$, und auf der rechten $-\frac 12$).
Diese Art von Fehler ist so beliebt, dass er sogar einen Namen hat, nämlich \emph{Bejahung des Konsequens}. 

Da logische Fehler einfach und schnell passieren, wollen wir feststellen, wie wir die Gültigkeit eines Arguments überprüfen
können. Dazu betrachten wir nicht notwendigerweise mathematische, sondern auch alltägliche Argumente. Ein Beispiel ist das
Folgende:

\begin{tabular}{l}
  Es ist Benzin im Tank oder der Motor läuft nicht.\\
  Der Motor läuft.\\
  \hline
  Benzin ist im Tank.
\end{tabular}

Die zwei Sätze über dem horizontalen Strich sind die sogenannten \emph{Prämissen}: Wir nehmen an, dass die Prämissen wahre
Aussagen sind. Ob das tatsächlich der Fall ist, soll uns im Rahmen der logischen Untersuchungen nicht interessieren; allerdings
ist es in der Praxis natürlich wichtig, dass die Prämissen tatsächlich wahr sind. Der Satz unter dem Strich heißt
\emph{Konklusion}: Das ist eine Aussage, deren Wahrheit mit Hilfe der Prämissen bewiesen werden soll. Man kann den Strich als
eines der Wörter "`also"' oder "`folglich"' interpretieren.

Es gibt zwei wichtige Beobachtungen bei der Untersuchung solcher Argumente. Erstens gibt es Wörter, die eine Information
ausdrücken; sie kommen üblicherweise in einem Argument mehrfach vor. Zweitens gibt es Wörter, die die Informationen verknüpfen:
sogenannte \emph{logische Ausdrücke}. Färbt man in dem obigen Argumente gleiche Informationen mit der gleichen Farbe und logische
Ausdrücke mit \textcolor{red}{rot} ein, dann ergibt sich:

\begin{tabular}{l}
  \textcolor{green}{Es ist Benzin im Tank} \textcolor{red}{oder} \textcolor{brown}{der Motor läuft} \textcolor{red}{nicht}.\\
  \textcolor{brown}{Der Motor läuft}.\\
  \arrayrulecolor{red}\hline
  \textcolor{green}{Benzin ist im Tank}.
\end{tabular}

Betrachten wir die zwei folgenden Argumente:

\begin{tabular}{l}
  \textcolor{green}{Bello ist ein Hund} \textcolor{red}{oder} \textcolor{brown}{er bellt} \textcolor{red}{nicht}.\\
  \textcolor{brown}{Bello bellt}.\\
  \arrayrulecolor{red}\hline
  \textcolor{green}{Bello ist ein Hund}.
\end{tabular}

\begin{tabular}{l}
  \textcolor{green}{Das reguläre Vieleck A hat drei bis sechs Ecken} \textcolor{red}{oder} 
  \textcolor{brown}{sein Innenwinkel ist \textcolor{red}{nicht} ein Teiler von $360^\circ$}.\\
  \textcolor{brown}{Der Innenwinkel von A ist ein Teiler von $360^\circ$}.\\
  \arrayrulecolor{red}\hline
  \textcolor{green}{A hat drei bis sechs Ecken}.
\end{tabular}

Wenn man die drei bisherigen Argumente anschaut, sieht man, dass alle eine Gemeinsamkeit haben: Sie sind alle drei von der Form

\begin{tabular}{l}
  \textcolor{green}{Grün} \textcolor{red}{oder} \textcolor{brown}{braun} \textcolor{red}{nicht}.\\
  \textcolor{brown}{Braun}.\\
  \arrayrulecolor{red}\hline
  \textcolor{green}{Grün}.
\end{tabular}

Wir sagen, sie haben \emph{dieselbe logische Form}.

\begin{aufgabe}
  Finde noch zwei weitere Argumente mit derselben logischen Form wie die Argumente bisher.
\end{aufgabe}

Wir wollen konkretisieren, was eine logische Form ist. Dazu gibt es vier Schritte, die man bei einem beliebigen Argument
durchführen kann. Diese Schritte will ich bei dem Argument mit Bello verdeutlichen.

Im ersten Schritt werden \emph{gleiche Argumente mit dem exakt gleichen Wortlaut ausgedrückt} (soweit das grammatikalisch möglich
ist). In dem Bello-Argument bedeutet das, dass "`er bellt"' in der ersten Prämisse durch den Wortlaut "`Bello bellt"' ersetzt
wird:

\begin{tabular}{l}
  \textcolor{green}{Bello ist ein Hund} \textcolor{red}{oder} \textcolor{brown}{Bello bellt} \textcolor{red}{nicht}.\\
  \textcolor{brown}{Bello bellt}.\\
  \arrayrulecolor{red}\hline
  \textcolor{green}{Bello ist ein Hund}.
\end{tabular}

Im zweiten Schritt \emph{werden Informationen durch Satzkonstanten abgekürzt}. Das bedeutet, dass wir die grünen Aussage "`Bello
ist ein Hund"' mit dem Buchstaben $A$ abkürzen und die Aussage "`Bello bellt"' mit $B$:

\begin{tabular}{l}
  \textcolor{green}{$A:=\text{"`Bello ist ein Hund"'}$}, \textcolor{brown}{$B:=\text{"`Bello bellt"'}$}\\[1ex]
  \textcolor{green}{$A$} \textcolor{red}{oder} \textcolor{brown}{$B$} \textcolor{red}{nicht}.\\
  \textcolor{brown}{$B$}.\\
  \arrayrulecolor{red}\hline
  \textcolor{green}{$A$}.
\end{tabular}

Das Zeichen $:=$ bedeutet dabei, dass die linke Seite durch die rechte definiert wird. Dagegen würde $=:$ bedeuten, dass die Seite
rechts vom Gleichheitszeichen durch das definiert wird, was links steht.

Im dritten Schritt werden die \emph{Satzkonstanten durch Satzvariablen} ersetzt:

\begin{tabular}{l}
  \textcolor{green}{$p$} \textcolor{red}{oder} \textcolor{brown}{$q$} \textcolor{red}{nicht}.\\
  \textcolor{brown}{$q$}.\\
  \arrayrulecolor{red}\hline
  \textcolor{green}{$p$}.
\end{tabular}

Die Zeichen \textcolor{green}{$p$} und \textcolor{red}{$q$} stehen jetzt nicht mehr für eine \emph{konkrete} Aussage, sondern
für eine \emph{beliebige} Aussage. Man könnte also jetzt \textcolor{red}{$p$} überall durch die Aussage "`Benzin ist im Tank"'
und \textcolor{brown}{$q$} durch die Aussage "`Der Motor läuft"' ersetzen und würde unser erstes Argument erhalten.

Im vierten Schritt werden jetzt noch die \emph{logischen Ausdrücke durch Symbole ersetzt}: 

\begin{tabular}{l}
  \textcolor{green}{$p$} \textcolor{red}{$\vee$} \textcolor{red}{$\neg$}\textcolor{brown}{$q$}.\\
  \textcolor{brown}{$q$}.\\
  \arrayrulecolor{red}\hline
  \textcolor{green}{$p$}.
\end{tabular}

Dabei kann man die Symbole für die logischen Ausdrücke aus folgender Tabelle entnehmen:

\begin{center}
  \begin{tabular}{rl}
    \toprule
    logischer Ausdruck&Symbol\\
    \midrule
    nicht $p$ bzw. $p$ nicht&$\neg p$\\
    $p$ oder $q$&$p\vee q$\\
    $p$ und $q$&$p\wedge q$\\
    wenn $p$, dann $q$&$p\rightarrow q$\\
    $p$ genau dann, wenn $q$&$p\leftrightarrow q$\\
    \bottomrule
  \end{tabular}
\end{center}

\begin{aufgabe}
  Führe die vier Schritte bei folgendem Argument durch:

  \begin{tabular}{l}
    Willi nimmt die Treppen oder er nimmt den Aufzug zu seiner Wohnung.\\
    Wenn er die Treppen nimmt, wird er oben müde sein.\\
    Wenn er den Aufzug nimmt, wird er den Beginn seiner Lieblingsserie verpassen.\\
    \hline
    Willi wird oben müde sein oder den Beginn seiner Lieblingsserie verpassen.
  \end{tabular}

\end{aufgabe}
\begin{proof}[Lösung]
  Beim ersten Schritt wird "`er nimmt den Aufzug zu seiner Wohnung"' zu "`er nimmt den Aufzug"', "`Willi nimmt die Treppen"' zu
  "`er nimmt die Treppen"' und "`Willi wird oben müde sein"' zu "`er wird oben müde sein"'. Außerdem wird in der Konklusion noch
  "`er wird"' eingefügt. Das Wort "`dann"', das bisher nicht explizit vorkam, wird ebenfalls eingefügt.

  \begin{tabular}{l}
    Er nimmt die Treppen oder er nimmt den Aufzug.\\
    Wenn er die Treppen nimmt, dann wird er oben müde sein.\\
    Wenn er den Aufzug nimmt, dann wird er den Beginn seiner Lieblingsserie verpassen.\\
    \hline
    Er wird oben müde sein oder er wird den Beginn seiner Lieblingsserie verpassen.
  \end{tabular}

  Im zweiten Schritt erhalten wir

  \begin{tabular}{l}
    $A:=\text{"`Er nimmt die Treppen"'}$, $B:=\text{"`Er nimmt den Aufzug"'}$,\\
    $C:=\text{"`Er wird oben müde sein"'}$, $D:=\text{"`Er wird den Beginn seiner Lieblingsserie verpassen"'}$.\\[1ex]
    $A$ oder $B$.\\
    Wenn $A$, dann $C$.\\
    Wenn $B$, dann $D$.\\
    \hline
    $C$ oder $D$.
  \end{tabular}

  Der dritte Schritt ergibt

  \begin{tabular}{l}
    $p$ oder $q$.\\
    Wenn $p$, dann $r$.\\
    Wenn $q$, dann $s$.\\
    \hline
    $r$ oder $s$.
  \end{tabular}

  Im vierten Schritt schließlich erhalten wir die logische Form

  \begin{tabular}{l}
    $p\vee q$.\\
    $p\rightarrow r$\\
    $q\rightarrow s$\\
    \hline
    $r\vee s$.
  \end{tabular}\renewcommand{\qedsymbol}{}
\end{proof}

\begin{aufgabe}
  Finde ein Argument mit einer weiteren logischen Form.
\end{aufgabe}

Wie können wir jetzt herausfinden, ob beispielsweise die logische Form

\begin{tabular}{l}
  $p\vee q$.\\
  $p\rightarrow r$\\
  $q\rightarrow s$\\
  \hline
  $r\vee s$.
\end{tabular}

ein gültiges Argument ergibt? Dazu nehmen wir an, dass alle Aussagen, die wir untersuchen wollen, entweder wahr oder falsch (und
nicht beides gleichzeitig) sind. Dann können wir die Bedeutung der logischen Ausdrücke präzisieren: Beispielsweise ist die Aussage
$\neg p$ ("`nicht $p$"') wahr, wenn $p$ falsch ist, und falsch, wenn $p$ war ist. Die Aussage $p\wedge q$ ("`$p$ und $q$"') ist
wahr, wenn $p$ und $q$ beide wahr sind, und ansonsten falsch. Das lässt sich durch \emph{Wahrheitstafeln} folgendermaßen
darstellen:

\begin{center}
  \begin{tabular}[t]{>{$}c<{$}||>{$}c<{$}}
    p&\neg p\\
    \hline
    w&f\\
    f&w
  \end{tabular}
  \hskip 2em
  \begin{tabular}[t]{>{$}c<{$}|>{$}c<{$}||>{$}c<{$}}
    p&q&p\wedge q\\
    \hline
    w&w&w\\
    w&f&f\\
    f&w&f\\
    f&f&f
  \end{tabular}
\end{center}

Auf der linken Seite des Doppelstrichs werden alle vorkommenden Aussagenvariablen aufgeführt. Jede kann unabhängig von den anderen
wahr oder falsch sein. Dabei wird "`wahr"' durch $w$ und "`falsch"' durch $f$ abgekürzt.

\begin{aufgabe}
  Stelle die Wahrheitstafeln für $\vee$, $\rightarrow$ und $\leftrightarrow$ auf.
\end{aufgabe}
\begin{proof}[Lösung]\renewcommand{\qedsymbol}{}\
  \begin{center}
    \begin{tabular}[t]{>{$}c<{$}|>{$}c<{$}||>{$}c<{$}}
      p&q&p\vee q\\
      \hline
      w&w&w\\
      w&f&w\\
      f&w&w\\
      f&f&f
    \end{tabular}
    \hskip 2em
    \begin{tabular}[t]{>{$}c<{$}|>{$}c<{$}||>{$}c<{$}}
      p&q&p\rightarrow q\\
      \hline
      w&w&w\\
      w&f&f\\
      f&w&w\\
      f&f&w
    \end{tabular}
    \hskip 2em
    \begin{tabular}[t]{>{$}c<{$}|>{$}c<{$}||>{$}c<{$}}
      p&q&p\leftrightarrow q\\
      \hline
      w&w&w\\
      w&f&f\\
      f&w&f\\
      f&f&w
    \end{tabular}
  \end{center}

  Das präzisiert die Bedeutung von "`oder"', "`wenn, dann"' und "`genau dann, wenn"'. Die Aussage "`$p$ oder $q$"' ist nach der
  ersten Tabelle genau dann wahr, wenn mindestens eine der Aussagen $p$ oder $q$ wahr ist (also auch, wenn beide wahr sind). Die
  Aussage "`wenn $p$, dann $q$"' ist nach der mittleren Tabelle immer wahr, außer wenn $p$ falsch und $q$ trotzdem wahr ist. 
  Schließlich ist die Aussage "`$p$ genau dann, wenn $q$"' wahr, wenn $p$ und $q$ beide wahr oder beide falsch sind.
\end{proof}

Jetzt sind wir so weit, dass wir Argumentationsformen auf deduktive Korrektheit, also Korrektheit im Sinne der Logik, überprüfen
können. Zunächst überprüfen wir die Korrektheit unseres Bello-Arguments vom Anfang, also des Arguments der Form

\begin{tabular}{>{$}l<{$}}
  p\vee (\neg q)\\
  q\\
  \hline
  p
\end{tabular}

Dazu stellen wir eine Wahrheitstafel mit allen vorkommenden logischen Ausdrücken auf, also mit $p$, $q$ und $p\vee(\neg q)$. Der
letzte Ausdruck ist selbst noch einmal zusammengesetzt, sodass wir auch noch $\neg q$ zur Analyse brauchen. Zusammen erhalten wir
die folgende Wahrheitstafel:

\begin{center}
  \begin{tabular}{>{$}c<{$}|>{$}c<{$}||>{$}c<{$}|>{$}c<{$}}
    p&q&\neg q&p\vee(\neg q)\\
    \hline
    w&w&f&w\\
    w&f&w&w\\
    f&w&f&f\\
    f&f&w&w
  \end{tabular}
\end{center}

Da wir annehmen, dass die Prämissen wahr sind, können wir alle Zeilen, bei denen $p\vee(\neg q)$ oder $q$ falsch ist,
wegstreichen. Übrig bleibt nur noch eine Zeile:

\newcommand{\tikzmark}[1]{\tikz[remember picture,overlay, baseline=-0.5ex]\node (#1){};}
\newcommand{\connect}[3][3mm]{\tikz[remember picture,overlay]\draw[shorten <=-#1, shorten >=-#1,red](#2)--(#3);}

\begin{center}
  \begin{tabular}{>{$}c<{$}|>{$}c<{$}||>{$}c<{$}|>{$}c<{$}}
    p&q&\neg q&p\vee(\neg q)\\
    \hline
    \colorbox{green}{$w$}&w&f&w\\
    \tikzmark{p1}w&\colorbox{red}{$f$}&w&w\tikzmark{p2}\\
    \tikzmark{p3}f&w&f&\colorbox{red}{$f$}\tikzmark{p4}\\
    \tikzmark{p5}f&\colorbox{red}{$f$}&w&w\tikzmark{p6}
  \end{tabular}
  \connect{p1}{p2}
  \connect{p3}{p4}
  \connect{p5}{p6}
\end{center}

In dieser Zeile ist die Konklusion $p$ allerdings wahr, sodass das Argument gültig ist: Wenn die Prämissen wahr sind, dann ist auf
jeden Fall auch die Konklusion wahr.

Ein weiteres Beispiel ist folgende Argumentationsform:

\begin{tabular}{>{$}l<{$}}
  p\rightarrow q\\
  q\rightarrow r\\
  \hline
  p\rightarrow r
\end{tabular}

Die fertige Wahrheitstabelle schaut folgendermaßen aus:

\begin{center}
  \begin{tabular}{>{$}c<{$}|>{$}c<{$}|>{$}c<{$}||>{$}c<{$}|>{$}c<{$}|>{$}c<{$}}
    p&q&r&p\rightarrow q&q\rightarrow r&p\rightarrow r\\
    \hline
    w&w&w&w&w&\colorbox{green}{$w$}\\
    \tikzmark{q1}w&w&f&w&\colorbox{red}{$f$}&f\tikzmark{q2}\\
    \tikzmark{q3}w&f&w&\colorbox{red}{$f$}&w&w\tikzmark{q4}\\
    \tikzmark{q5}w&f&f&\colorbox{red}{$f$}&w&f\tikzmark{q6}\\
    f&w&w&w&w&\colorbox{green}{$w$}\\
    \tikzmark{q7}f&w&f&w&\colorbox{red}{$f$}&w\tikzmark{q8}\\
    f&f&w&w&w&\colorbox{green}{$w$}\\
    f&f&f&w&w&\colorbox{green}{$w$}
    \connect{q1}{q2}
    \connect{q3}{q4}
    \connect{q5}{q6}
    \connect{q7}{q8}
  \end{tabular}
\end{center}

Auch hier ist die Konklusion wieder in allen übriggebliebenen Zeilen wahr und das Argument daher gültig. Es sagt aus, dass
der logische Ausdruck "`$\rightarrow$"' \emph{transitiv} ist, dass man also aus Zwischenschritten bestehende Implikationen 
zusammensetzen kann.

\begin{center}
\end{center}

\pagebreak
\begin{aufgabe}
  Überprüfe folgende Argumentationsformen auf deduktive Korrektheit:

  \begin{tabular}{ll >{\hskip 1em}ll >{\hskip 1em}ll}
    a)&\textbf{Modus ponens}&b)&\textbf{Modus tollens}&c)&\textbf{Verneinung des Antezedens}\\
      &$p\rightarrow q$  & &  $p\rightarrow q$  & &  $p\rightarrow q$\\
      &$p$		 & &  $\neg q$		& &  $\neg p$\\
    \hline
      &$q$		 & &  $\neg p$		& &  $\neg q$
  \end{tabular}\\[1ex]

  \begin{tabular}{ll >{\hskip 1em}ll}
    d)&\textbf{Disjunktiver Syllogismus}&e)&\textbf{Bejahung des Konsequens}\\
      &$p\vee q$  & &  $p\rightarrow q$\\
      &$\neg p$	  & &  $q$\\
    \hline
      &$q$	  & &  $p$
  \end{tabular}\\[1ex]

  \begin{tabular}{ll}
    f)&\textbf{Reductio ad absurdum}\\
      &$\neg p\rightarrow(q\wedge\neg q)$\\
    \hline
      &$p$
  \end{tabular}
\end{aufgabe}

Es bleibt zu sagen, dass die Annahme, dass alle Aussagen entweder wahr oder falsch sind, ein Axiom sind, das nicht weiter
begründet werden kann. Tatsächlich gibt es auch Formen von formaler Logik, die diese Annahme nicht verwenden; man sagt auch, dass
der \emph{Satz vom ausgeschlossenen Dritten} oder das \emph{tertium non datur} in diesen Arten von Logik nicht gilt.

Unabhängig davon haben wir nur die sogenannte \emph{Aussagenlogik} untersucht. In der \emph{Prädikatenlogik} werden Aussagen der
Art "`Alle Hunde sind Säugetiere"' genauer untersucht: Der Gegenstand der Prädikatenlogik sind Aussagen, die für \emph{alle} oder
\emph{einige} Objekte eines bestimmten Typs erfüllt sind.

\end{document}
