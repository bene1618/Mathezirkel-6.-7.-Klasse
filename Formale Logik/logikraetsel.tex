\documentclass{zusammenfassung}
\graphicspath{ {./illustrationen/} }

\begin{document}
\maketitle{Klasse 6/7}{13. Januar 2015}{2014/2015}

\begin{aufgabe}[Albert Schweitzer in Lambarene]
Urwalddoktor Albert Schweitzer hat es nach diesem Rätsel in seiner Klinik nur mit folgenden zwei Volksstämmen zu tun:

\begin{itemize}
  \item mit den Wahrowambos (sie können immer nur die Wahrheit sprechen) und
  \item mit den Lügowambos (sie können immer nur lügen).
\end{itemize}

Als Herr Dr. Schweitzer eines Morgens seinen Wartesaal betritt, bemerkt er drei neue Gesichter, die er bisher noch nie gesehen 
hatte. Der Arzt fragt den ersten davon: "`Zu welchem der beiden Stämme gehörst Du?"'
Der Gefragte murmelt etwas völlig Unverständliches, worauf sich der zweite Neue zu Wort meldet und sagt: 
"`Herr Doktor, der erste hat gesagt, dass er ein Wahrowambo sei, und das stimmt, denn auch ich bin Wahrowambo und kenne ihn 
persönlich."'
Da protestiert der dritte Neuzugang sofort und schreit: "`Falsch, Herr Schweitzer, ich bin hier der einzige Wahrowambo und die 
beiden anderen sind Lügowambos!"'

Diese Aussagen genügen Albert Schweitzer, um die drei Neuzugänge mit absoluter Sicherheit den beiden dort ansässigen 
Stammesgruppen zuzuordnen. Welche Logik wendet Albert Schweitzer an?
\end{aufgabe}

%\begin{proof}[Lösung]
%Angenommen, der dritte wäre tatsächlich ein Wahrowambo. Dann hätte er die Wahrheit gesagt, also wären die ersten beiden Lügowambos
%und hätten beide gelogen. Der erste hätte dann (fälschlicherweise) gesagt, dass er ein Wahrowambo sei, und dann hätte der zweite
%(ebenfalls fälschlicherweise) behaupten müssen, dass der erste gesagt hätte, er sei ein Lügowambo. Das ist nicht geschehen, also
%ist die Annahme, dass der dritte ein Wahrowambo ist, falsch.
%
%Damit muss auch die Behauptung des dritten (jetzt als Lügowambo enttarnten) Menschen falsch sein, dass es sich bei den ersten
%beiden um Lügowambos handeln würde. Es folgt, dass die ersten beiden Wahrowambos gewesen sein müssen.
%\end{proof}

\begin{aufgabe}[Vier Freunde]
Es waren einmal vier Freunde, die sich uneinig waren, wer von ihnen im Recht ist:

\begin{itemize}
  \item Thomas sagte: "`Genau zwei von uns haben recht."'
  \item "`So eng würde ich das jetzt nicht sehen"', sagte Annika, "`denn mindestens zwei von uns liegen falsch."'
  \item "`Also, was Annika da behauptet, ist auf jeden Fall falsch"', entgegnete Mark.
  \item Daraufhin meine Lisa: "`Höchstens einer von uns liegt richtig."'
\end{itemize}

Wer hat denn nun recht?
\end{aufgabe}
\enlargethispage{3cm}

\begin{aufgabe}[Unbeantwortbare Frage]
  Ein Mann sagt zu einem anderen:

  "`Ich werde Dir eine Frage stellen, auf die es eine eindeutig richtige Antwort gibt -- entweder ja oder nein --, aber es wird Dir
  unmöglich sein, meine Frage zu beantworten. Möglicherweise wirst du die richtige Antwort kennen, aber du wirst sie mir nicht
  geben. Jeder andere wäre vielleicht in der Lage, die Antwort zu liefern, du aber nicht."'

  Welche Frage wird er ihm stellen?
\end{aufgabe}

\begin{aufgabe}[Wer lügt?]
  Fünf Personen A, B, C, D und E unterhalten sich:
  
  \begin{itemize}
    \item A: B lügt dann und nur dann, wenn D die Wahrheit sagt.
    \item B: Wenn C die Wahrheit sagt, dann ist A oder D ein Lügner.
    \item C: E lügt, und auch A oder B lügen.
    \item D: Wenn B die Wahrheit sagt, dann auch A oder C.
    \item E: Unter den Personen A, C und D befindet sich mindestens ein Lügner.
  \end{itemize}
  
  Genau zwei Personen lügen. Welche?
\end{aufgabe}

\begin{aufgabe}[Einstein-Rätsel]
Fünf Häuser stehen nebeneinander. In ihnen wohnen Menschen von fünf unterschiedlichen Nationalitäten, die 
fünf unterschiedliche Getränke trinken, fünf unterschiedliche Zigarettenmarken rauchen und fünf unterschiedliche Haustiere haben.

\begin{itemize}
  \item Der Brite lebt im roten Haus.
  \item Der Schwede hält sich einen Hund.
  \item Der Däne trinkt gern Tee.
  \item Das grüne Haus steht (direkt) links neben dem weißen Haus.
  \item Der Besitzer des grünen Hauses trinkt Kaffee.
  \item Die Person, die Pall Mall raucht, hat einen Vogel.
  \item Der Mann im mittleren Haus trinkt Milch.
  \item Der Bewohner des gelben Hauses raucht Dunhill.
  \item Der Norweger lebt im ersten Haus.
  \item Der Marlboro-Raucher wohnt neben der Person mit der Katze.
  \item Der Mann mit dem Pferd lebt neben der Person, die Dunhill raucht.
  \item Der Winfield-Raucher trinkt gern Bier.
  \item Der Norweger wohnt neben dem blauen Haus.
  \item Der Deutsche raucht Rothmanns.
  \item Der Marlboro-Raucher hat einen Nachbarn, der Wasser trinkt.
\end{itemize}

Wem gehört der Fisch?
\end{aufgabe}

\end{document}
